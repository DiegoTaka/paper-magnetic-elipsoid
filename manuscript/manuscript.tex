%% Copernicus Publications Manuscript Preparation Template for LaTeX Submissions
%% ---------------------------------
%% This template should be used for copernicus.cls
%% The class file and some style files are bundled in the Copernicus Latex Package which can be downloaded from the different journal webpages.
%% For further assistance please contact the Copernicus Publications at: publications@copernicus.org
%% http://publications.copernicus.org


%% Please use the following documentclass and Journal Abbreviations for Discussion Papers and Final Revised Papers.


%% 2-Column Papers and Discussion Papers
\documentclass[gmd, manuscript]{copernicus}



%% Journal Abbreviations (Please use the same for Discussion Papers and Final Revised Papers)

% Archives Animal Breeding (aab)
% Atmospheric Chemistry and Physics (acp)
% Advances in Geosciences (adgeo)
% Advances in Statistical Climatology, Meteorology and Oceanography (ascmo)
% Annales Geophysicae (angeo)
% ASTRA Proceedings (ap)
% Atmospheric Measurement Techniques (amt)
% Advances in Radio Science (ars)
% Advances in Science and Research (asr)
% Biogeosciences (bg)
% Climate of the Past (cp)
% Drinking Water Engineering and Science (dwes)
% Earth System Dynamics (esd)
% Earth Surface Dynamics (esurf)
% Earth System Science Data (essd)
% Fossil Record (fr)
% Geographica Helvetica (gh)
% Geoscientific Instrumentation, Methods and Data Systems (gi)
% Geoscientific Model Development (gmd)
% Geothermal Energy Science (gtes)
% Hydrology and Earth System Sciences (hess)
% History of Geo- and Space Sciences (hgss)
% Journal of Sensors and Sensor Systems (jsss)
% Mechanical Sciences (ms)
% Natural Hazards and Earth System Sciences (nhess)
% Nonlinear Processes in Geophysics (npg)
% Ocean Science (os)
% Proceedings of the International Association of Hydrological Sciences (piahs)
% Primate Biology (pb)
% Scientific Drilling (sd)
% SOIL (soil)
% Solid Earth (se)
% The Cryosphere (tc)
% Web Ecology (we)
% Wind Energy Science (wes)


%% \usepackage commands included in the copernicus.cls:
%\usepackage[german, english]{babel}
%\usepackage{tabularx}
%\usepackage{cancel}
%\usepackage{multirow}
%\usepackage{supertabular}
%\usepackage{algorithmic}
%\usepackage{algorithm}
%\usepackage{amsthm}
%\usepackage{float}
%\usepackage{subfig}
%\usepackage{rotating}


\begin{document}

\title{3D MAGNETIC MODELLING FOR ELLIPSOIDS}


% \Author[affil]{given_name}{surname}

\Author[1]{Diego}{Takahashi Tomazella}
\Author[1]{Vanderlei}{C. Oliveira Jr}

\affil[1]{Department of Geophysics, Observatorio Nacional, Rio de Janeiro, Brazil}
%\affil[]{}

%% The [] brackets identify the author with the corresponding affiliation. 1, 2, 3, etc. should be inserted.



\runningtitle{TEXT}

\runningauthor{TEXT}

\correspondence{DIEGO TAKAHASHI TOMAZELLA (diego.takahashi@gmail.com)}



\received{}
\pubdiscuss{} %% only important for two-stage journals
\revised{}
\accepted{}
\published{}

%% These dates will be inserted by Copernicus Publications during the typesetting process.


\firstpage{1}

\maketitle



\begin{abstract}

TEXT

\end{abstract}



\introduction  %% \introduction[modified heading if necessary]

TEXT

\section{Methodology}


\subsection{Geometrical aspects}

Let $(x, y, z)$ be a point referred to a Cartesian coordinate system with axes
$x$, $y$ and $z$ pointing to, respectively, North, East and down.

Consider an ellipsoidal body with centre at the point $(x_{c}, y_{c}, z_{c})$,
semi-axes defined by positive constants $a$, $b$, $c$, and orientation defined by
three angles $\alpha$, $\beta$, and $\gamma$.

The points $(x, y, z)$ located on the surface of this ellipsoidal body
satisfy the following equation:
\begin{equation}
(\mathbf{r} - \mathbf{r}_c)^T \mathbf{A} (\mathbf{r} - \mathbf{r}_c) = 1 \: ,
\label{eq:ellipsoid_surface}
\end{equation}
where $\mathbf{r} = [\begin{array}{ccc} x & y & z \end{array} ]^{\top}$,
$\mathbf{r}_{c} = [\begin{array}{ccc} x_{c} & y_{c} & z_{c} \end{array} ]^{\top}$,
$\mathbf{A}$ is a positive definite matrix given by
\begin{equation}
\mathbf{A} = \mathbf{V}
\left[ \begin{array}{ccc}
a^{-2} & 0 & 0 \\
0 & b^{-2} & 0 \\
0 & 0 & c^{-2} 
\end{array} \right] \mathbf{V}^{\top} \: ,
\label{eq:A}
\end{equation}
and $\mathbf{V}$ is an orthogonal matrix whose columns are defined
by unit vectors $\mathbf{v}_{1}$, $\mathbf{v}_{2}$, and $\mathbf{v}_{3}$.

The vectors $\mathbf{v}_{1}$, $\mathbf{v}_{2}$, and $\mathbf{v}_{3}$ 
have the same direction as the semi-axes $a$, $b$, $c$ of the 
ellipsoid, depend on the orientation angles 
$\alpha$, $\beta$, $\gamma$ and are defined 
according to the ellipsoid type.

For triaxial ellipsoids (i.e., $a > b > c$), the vectors 
$\mathbf{v}_{1}$, $\mathbf{v}_{2}$, and $\mathbf{v}_{3}$ are given
by \citep{clark1986magnetic}:
\begin{equation}
\mathbf{v}_{1} = \left[\begin{array}{c} 
                  -\cos\alpha \; \cos\delta \\
                  -\sin\alpha \; \cos\delta \\
                  -\sin\delta
                  \end{array} \right] \: ,
\label{eq:v1_triaxial_prolate}
\end{equation}
\begin{equation}
\mathbf{v}_{2} = \left[\begin{array}{c} 
                  \cos\alpha \; \cos\gamma \; \sin\delta + \sin\alpha \; \sin\gamma \\                   
                  \sin\alpha \; \cos\gamma \; \sin\delta - \cos\alpha \; \sin\gamma \\ 
                  -\cos\gamma \; \cos\delta
                  \end{array} \right] \: ,
\label{eq:v2_triaxial_prolate}
\end{equation}                   
\begin{equation}                    
\mathbf{v}_{3} = \left[\begin{array}{c} 
                  \sin\alpha \; \cos\gamma - \cos\alpha \; \sin\gamma \; \sin\delta \\                    
                  -\cos\alpha \; \cos\gamma - \sin\alpha \; \sin\gamma \; \sin\delta \\
                  \sin\gamma \; \cos\delta
                  \end{array} \right] \: .
\label{eq:v3_triaxial_prolate}
\end{equation}

Similarly, the vectors $\mathbf{v}_{1}$, $\mathbf{v}_{2}$, and $\mathbf{v}_{3}$
defining the semi-axes of prolate ellipsoids (i.e., $a > b = c$) are 
calculated by using equations \ref{eq:v1_triaxial_prolate}, \ref{eq:v2_triaxial_prolate}, 
and \ref{eq:v3_triaxial_prolate}, but with $\gamma = 0^{\circ}$ \citep{emerson1985magnetic}.

Finally, the vectors $\mathbf{v}_{1}$, $\mathbf{v}_{2}$, and $\mathbf{v}_{3}$
defining the semi-axes of oblate ellipsoids (i.e., $a < b = c$) are

PAREI AQUI



%whose eigenvectors define the 
%axes of the ellipsoid and the eigenvalues of A are the reciprocals of the squares of the semi-axes


\subsection{TEXT}

\subsubsection{TEXT}
TEXT

\conclusions  %% \conclusions[modified heading if necessary]
TEXT




\appendix
\section{}    %% Appendix A

\subsection{}                               %% Appendix A1, A2, etc.


\authorcontribution{TEXT}

\begin{acknowledgements}
TEXT
\end{acknowledgements}


%% REFERENCES

%% The reference list is compiled as follows:

%%\begin{thebibliography}{}

%%\bibitem[AUTHOR(YEAR)]{LABEL}
%%REFERENCE 1

%%\bibitem[AUTHOR(YEAR)]{LABEL}
%%REFERENCE 2

%%\end{thebibliography}

\bibliographystyle{copernicus}
\bibliography{references}

%% Since the Copernicus LaTeX package includes the BibTeX style file copernicus.bst,
%% authors experienced with BibTeX only have to include the following two lines:
%%
\bibliographystyle{copernicus}
\bibliography{references.bib}
%%
%% URLs and DOIs can be entered in your BibTeX file as:
%%
%% URL = {http://www.xyz.org/~jones/idx_g.htm}
%% DOI = {10.5194/xyz}


%% LITERATURE CITATIONS
%%
%% command                        & example result
%% \citet{jones90}|               & Jones et al. (1990)
%% \citep{jones90}|               & (Jones et al., 1990)
%% \citep{jones90,jones93}|       & (Jones et al., 1990, 1993)
%% \citep[p.~32]{jones90}|        & (Jones et al., 1990, p.~32)
%% \citep[e.g.,][]{jones90}|      & (e.g., Jones et al., 1990)
%% \citep[e.g.,][p.~32]{jones90}| & (e.g., Jones et al., 1990, p.~32)
%% \citeauthor{jones90}|          & Jones et al.
%% \citeyear{jones90}|            & 1990



%% FIGURES

%% ONE-COLUMN FIGURES

%%f
%\begin{figure}[t]
%\includegraphics[width=8.3cm]{FILE NAME}
%\caption{TEXT}
%\end{figure}
%
%%% TWO-COLUMN FIGURES
%
%%f
%\begin{figure*}[t]
%\includegraphics[width=12cm]{FILE NAME}
%\caption{TEXT}
%\end{figure*}
%
%
%%% TABLES
%%%
%%% The different columns must be seperated with a & command and should
%%% end with \\ to identify the column brake.
%
%%% ONE-COLUMN TABLE
%
%%t
%\begin{table}[t]
%\caption{TEXT}
%\begin{tabular}{column = lcr}
%\tophline
%
%\middlehline
%
%\bottomhline
%\end{tabular}
%\belowtable{} % Table Footnotes
%\end{table}
%
%%% TWO-COLUMN TABLE
%
%%t
%\begin{table*}[t]
%\caption{TEXT}
%\begin{tabular}{column = lcr}
%\tophline
%
%\middlehline
%
%\bottomhline
%\end{tabular}
%\belowtable{} % Table Footnotes
%\end{table*}
%
%
%%% NUMBERING OF FIGURES AND TABLES
%%%
%%% If figures and tables must be numbered 1a, 1b, etc. the following command
%%% should be inserted before the begin{} command.
%
%\addtocounter{figure}{-1}\renewcommand{\thefigure}{\arabic{figure}a}
%
%
%%% MATHEMATICAL EXPRESSIONS
%
%%% All papers typeset by Copernicus Publications follow the math typesetting regulations
%%% given by the IUPAC Green Book (IUPAC: Quantities, Units and Symbols in Physical Chemistry,
%%% 2nd Edn., Blackwell Science, available at: http://old.iupac.org/publications/books/gbook/green_book_2ed.pdf, 1993).
%%%
%%% Physical quantities/variables are typeset in italic font (t for time, T for Temperature)
%%% Indices which are not defined are typeset in italic font (x, y, z, a, b, c)
%%% Items/objects which are defined are typeset in roman font (Car A, Car B)
%%% Descriptions/specifications which are defined by itself are typeset in roman font (abs, rel, ref, tot, net, ice)
%%% Abbreviations from 2 letters are typeset in roman font (RH, LAI)
%%% Vectors are identified in bold italic font using \vec{x}
%%% Matrices are identified in bold roman font
%%% Multiplication signs are typeset using the LaTeX commands \times (for vector products, grids, and exponential notations) or \cdot
%%% The character * should not be applied as mutliplication sign
%
%
%%% EQUATIONS
%
%%% Single-row equation
%
%\begin{equation}
%
%\end{equation}
%
%%% Multiline equation
%
%\begin{align}
%& 3 + 5 = 8\\
%& 3 + 5 = 8\\
%& 3 + 5 = 8
%\end{align}
%
%
%%% MATRICES
%
%\begin{matrix}
%x & y & z\\
%x & y & z\\
%x & y & z\\
%\end{matrix}
%
%
%%% ALGORITHM
%
%\begin{algorithm}
%\caption{�}
%\label{a1}
%\begin{algorithmic}
%�
%\end{algorithmic}
%\end{algorithm}
%
%
%%% CHEMICAL FORMULAS AND REACTIONS
%
%%% For formulas embedded in the text, please use \chem{}
%
%%% The reaction environment creates labels including the letter R, i.e. (R1), (R2), etc.
%
%\begin{reaction}
%%% \rightarrow should be used for normal (one-way) chemical reactions
%%% \rightleftharpoons should be used for equilibria
%%% \leftrightarrow should be used for resonance structures
%\end{reaction}
%
%
%%% PHYSICAL UNITS
%%%
%%% Please use \unit{} and apply the exponential notation


\end{document}
