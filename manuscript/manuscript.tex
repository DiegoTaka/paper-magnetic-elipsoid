%% Copernicus Publications Manuscript Preparation Template for LaTeX Submissions
%% ---------------------------------
%% This template should be used for copernicus.cls
%% The class file and some style files are bundled in the Copernicus Latex Package which can be downloaded from the different journal webpages.
%% For further assistance please contact the Copernicus Publications at: publications@copernicus.org
%% http://publications.copernicus.org


%% Please use the following documentclass and Journal Abbreviations for Discussion Papers and Final Revised Papers.


%% 2-Column Papers and Discussion Papers
\documentclass[gmd, manuscript]{copernicus}



%% Journal Abbreviations (Please use the same for Discussion Papers and Final Revised Papers)

% Archives Animal Breeding (aab)
% Atmospheric Chemistry and Physics (acp)
% Advances in Geosciences (adgeo)
% Advances in Statistical Climatology, Meteorology and Oceanography (ascmo)
% Annales Geophysicae (angeo)
% ASTRA Proceedings (ap)
% Atmospheric Measurement Techniques (amt)
% Advances in Radio Science (ars)
% Advances in Science and Research (asr)
% Biogeosciences (bg)
% Climate of the Past (cp)
% Drinking Water Engineering and Science (dwes)
% Earth System Dynamics (esd)
% Earth Surface Dynamics (esurf)
% Earth System Science Data (essd)
% Fossil Record (fr)
% Geographica Helvetica (gh)
% Geoscientific Instrumentation, Methods and Data Systems (gi)
% Geoscientific Model Development (gmd)
% Geothermal Energy Science (gtes)
% Hydrology and Earth System Sciences (hess)
% History of Geo- and Space Sciences (hgss)
% Journal of Sensors and Sensor Systems (jsss)
% Mechanical Sciences (ms)
% Natural Hazards and Earth System Sciences (nhess)
% Nonlinear Processes in Geophysics (npg)
% Ocean Science (os)
% Proceedings of the International Association of Hydrological Sciences (piahs)
% Primate Biology (pb)
% Scientific Drilling (sd)
% SOIL (soil)
% Solid Earth (se)
% The Cryosphere (tc)
% Web Ecology (we)
% Wind Energy Science (wes)


%% \usepackage commands included in the copernicus.cls:
%\usepackage[german, english]{babel}
%\usepackage{tabularx}
%\usepackage{cancel}
%\usepackage{multirow}
%\usepackage{supertabular}
%\usepackage{algorithmic}
%\usepackage{algorithm}
%\usepackage{amsthm}
%\usepackage{float}
%\usepackage{subfig}
%\usepackage{rotating}


\begin{document}

\title{3D Magnetic modelling of ellipsoidal bodies}


% \Author[affil]{given_name}{surname}

\Author[1]{Diego}{Takahashi Tomazella}
\Author[1]{Vanderlei}{C. Oliveira Jr}

\affil[1]{Department of Geophysics, Observat\'{o}rio Nacional, Rio de Janeiro, Brazil}
%\affil[]{}

%% The [] brackets identify the author with the corresponding affiliation. 1, 2, 3, etc. should be inserted.



\runningtitle{TEXT}

\runningauthor{TEXT}

\correspondence{Diego Takahashi Tomazella (diego.takahashi@gmail.com)}



\received{}
\pubdiscuss{} %% only important for two-stage journals
\revised{}
\accepted{}
\published{}

%% These dates will be inserted by Copernicus Publications during the typesetting process.


\firstpage{1}

\maketitle



\begin{abstract}

TEXT

\end{abstract}



\introduction  %% \introduction[modified heading if necessary]

TEXT

\section{Methodology}


\subsection{Geometrical parameters and coordinate systems}

Let $(x, y, z)$ be a point referred to a Cartesian coordinate system with axes
$x$, $y$ and $z$ pointing to, respectively, North, East and down.
For convenience, we denominate this coordinate system as 
\textit{main coordinate system}.
Let us consider an ellipsoidal body with centre at the point $(x_{c}, y_{c}, z_{c})$,
semi-axes defined by positive constants $a$, $b$, $c$, 
where $a > b > c$, and orientation defined by
three angles $\alpha$, $\beta$, and $\gamma$.
The points $(x, y, z)$ located on the surface of this ellipsoidal body
satisfy the following equation:
\begin{equation}
(\mathbf{r} - \mathbf{r}_c)^T \mathbf{A} (\mathbf{r} - \mathbf{r}_c) = 1 \: ,
\label{eq:ellipsoid_surface}
\end{equation}
where $\mathbf{r} = [\begin{array}{ccc} x & y & z \end{array} ]^{\top}$,
$\mathbf{r}_{c} = [\begin{array}{ccc} x_{c} & y_{c} & z_{c} \end{array} ]^{\top}$,
$\mathbf{A}$ is a positive definite matrix given by
\begin{equation}
\mathbf{A} = \mathbf{V}
\left[ \begin{array}{ccc}
a^{-2} & 0 & 0 \\
0 & b^{-2} & 0 \\
0 & 0 & c^{-2} 
\end{array} \right] \mathbf{V}^{\top} \: ,
\label{eq:A}
\end{equation}
and $\mathbf{V}$ is an orthogonal matrix whose columns are defined
by unit vectors $\mathbf{v}_{1}$, $\mathbf{v}_{2}$, and $\mathbf{v}_{3}$.

The vectors $\mathbf{v}_{1}$, $\mathbf{v}_{2}$, and $\mathbf{v}_{3}$ 
depend on the orientation angles 
$\alpha$, $\beta$, $\gamma$ and are defined 
as follows \citep{clark1986magnetic}:
\begin{equation}
\mathbf{v}_{1} = \left[\begin{array}{c} 
                  -\cos\alpha \; \cos\delta \\
                  -\sin\alpha \; \cos\delta \\
                  -\sin\delta
                  \end{array} \right] \: ,
\label{eq:v1_triaxial_prolate}
\end{equation}
\begin{equation}
\mathbf{v}_{2} = \left[\begin{array}{c} 
                  \cos\alpha \; \cos\gamma \; \sin\delta + \sin\alpha \; \sin\gamma \\                   
                  \sin\alpha \; \cos\gamma \; \sin\delta - \cos\alpha \; \sin\gamma \\ 
                  -\cos\gamma \; \cos\delta
                  \end{array} \right] \: ,
\label{eq:v2_triaxial_prolate}
\end{equation}                   
\begin{equation}                    
\mathbf{v}_{3} = \left[\begin{array}{c} 
                  \sin\alpha \; \cos\gamma - \cos\alpha \; \sin\gamma \; \sin\delta \\                    
                  -\cos\alpha \; \cos\gamma - \sin\alpha \; \sin\gamma \; \sin\delta \\
                  \sin\gamma \; \cos\delta
                  \end{array} \right] \: .
\label{eq:v3_triaxial_prolate}
\end{equation}
For triaxial ellipsoids (i.e., $a > b > c$), the orthogonal matrix
$\mathbf{V}$ (equation \ref{eq:A}) is calculated by using equations
\ref{eq:v1_triaxial_prolate}, \ref{eq:v2_triaxial_prolate}, and
\ref{eq:v3_triaxial_prolate} as follows:
\begin{equation}
\mathbf{V} = \left[ \begin{array}{ccc}
\mathbf{v}_{1} & \mathbf{v}_{2} & \mathbf{v}_{3}
\end{array} \right] \: .
\label{eq:V_triaxial_prolate}
\end{equation}
Similarly, the matrix $\mathbf{V}$ (equation \ref{eq:A}) for
prolate ellipsoids (i.e., $a > b = c$) is calculated according
to equation \ref{eq:V_triaxial_prolate} by using equations
\ref{eq:v1_triaxial_prolate}, \ref{eq:v2_triaxial_prolate}, and
\ref{eq:v3_triaxial_prolate}, but with $\gamma = 0^{\circ}$ 
\citep{emerson1985magnetic}.
Finally, the matrix $\mathbf{V}$ (equation \ref{eq:A}) for
oblate ellipsoids (i.e., $a < b = c$) is calculated by 
using equations \ref{eq:v1_triaxial_prolate}, \ref{eq:v2_triaxial_prolate}, and
\ref{eq:v3_triaxial_prolate}, with $\gamma = 0^{\circ}$, as follows
\citep{emerson1985magnetic}:
\begin{equation}
\mathbf{V} = \left[ \begin{array}{ccc}
\mathbf{v}_{2} & \mathbf{v}_{1} & -\mathbf{v}_{3}
\end{array} \right] \: .
\label{eq:V_oblate}
\end{equation}
The orientation of the semi-axes $a$, $b$, and $c$ are defined 
by the first, second, and third columns of the matrix 
$\mathbf{V}$ given by equation \ref{eq:V_triaxial_prolate},
in the case of a triaxial or prolate ellipsoid, or the
matrix $\mathbf{V}$ given by equation \ref{eq:V_oblate},
in the case of an oblate ellipsoid.

The magnetic modelling of an ellipsoidal body is commonly performed
in a particular Cartesian coordinate system that is aligned 
with the body semi-axes
and has the origin coincident with the body centre.
For convenience, we denominate this particular coordinate 
system as \textit{local coordinate system}.
The relationship between the Cartesian coordinates 
$(\tilde{x}, \tilde{y}, \tilde{z})$ of a point in 
a local coordinate system and the Cartesian 
coordinates $(x, y, z)$ of the same point in the main
system is given by:
\begin{equation}
\tilde{\mathbf{r}} = \mathbf{V}^{\top} \left( \mathbf{r} - \mathbf{r}_{c} \right) \: ,
\label{eq:coord_transformation}
\end{equation}
where 
$\tilde{\mathbf{r}} = [\begin{array}{ccc} \tilde{x} & 
                                          \tilde{y} & 
                                          \tilde{z} \end{array} ]^{\top}$,
$\mathbf{r}$ and $\mathbf{r}_{c}$
are defined in equation \ref{eq:ellipsoid_surface} and the matrix $\mathbf{V}$
is defined according to equations \ref{eq:V_triaxial_prolate} or
\ref{eq:V_oblate}, depending on the ellipsoid type.
Afterwards, we use the superscript "$\sim$" to implicitly
define quantities referred to a local coordinate system.

%A last but not least set of parameters defining the geometry of
%an ellipsoid are their focal points.
%The focal points are defined on the axes of the local coordinate 
%system as follows:
%\begin{equation}
%\left(\tilde{x}, \tilde{y}, \tilde{z} \right) = 
%\begin{cases}
%\left( \pm \sqrt{a^{2} - b^{2}}, 0, 0 \right) \\
%\left( \pm \sqrt{a^{2} - c^{2}}, 0, 0 \right) \\
%\left( 0, \pm \sqrt{b^{2} - c^{2}}, 0 \right)
%\end{cases} \: .
%\label{eq:focal-points}
%\end{equation}
%All ellipsoids sharing a common set of focal points form a family of confocal ellipsoids.

\subsection{Demagnetizing field within ellipsoids}

Consider a magnetized ellipsoid immersed in
a uniform magnetic field $\mathbf{H}_{0}$ (in $\unit{Am^{-1}}$).
The total magnetic intensity field $\mathbf{H}(\mathbf{r})$
at the position $\mathbf{r}$ (equations \ref{eq:A} and \ref{eq:coord_transformation}) 
of a point referred to the main coordinate system is defined
as follows
\begin{equation}
\mathbf{H}(\mathbf{r}) = \mathbf{H}_{0} - \nabla \phi(\mathbf{r}) \: ,
\label{eq:total-H-field}
\end{equation}
where the second term, representing the induced magnetic field,
is the negative gradient of the magnetic scalar potential 
$\phi(\mathbf{r})$ given by
\begin{equation}
\phi(\mathbf{r}) = \frac{1}{4\pi} \iiint_{V} 
\mathbf{J}(\mathbf{r}^{\prime})^{\top} \left(
\frac{\mathbf{r} - \mathbf{r}^{\prime}}{\| \mathbf{r} - \mathbf{r}^{\prime} \|^{3}}
\right) \, dx^{\prime}dy^{\prime}dz^{\prime} \: .
\label{eq:phi-potential}
\end{equation}
In this equation, $\mathbf{r}^{\prime} = [\begin{array}{ccc} 
x^{\prime} & y^{\prime} & z^{\prime} \end{array} ]^{\top}$
is the position vector of a point located within the volume $V$, 
the integral is conducted over the variables 
$x^{\prime}$, $y^{\prime}$ and, $z^{\prime}$ representing
the coordinates of a point located within the volume $V$
of the ellipsoid, $\| \cdot \|$ denotes the Euclidean norm and 
$\mathbf{J}(\mathbf{r}^{\prime})$ is the magnetization vector
(in $\unit{Am^{-1}}$).
Equation \ref{eq:total-H-field} is valid anywhere, 
independently if the position vector $\mathbf{r}$ represents
a point located inside or outside the magnetized body.

According to \citet{maxwell1873}, the only finite bodies that 
can be uniformly magnetized in the presence of a uniform and static 
magnetic field are the ones bounded by surfaces of second degree, which
are ellipsoids.

This property comes from the fact that the gravitational potential
within an ellipsoid with uniform density is a quadratic function
of the spatial coordinates.


Based on this, the magnetization vector 
$\mathbf{J}(\mathbf{r}^{\prime})$ (equation \ref{eq:phi-potential})
can be moved outside the integral, resulting that
%\begin{equation}
%
%\label{eq:total-H-field-matrix}
%\end{equation}


\subsection{Magnetic induction}




\subsection{TEXT}

\subsubsection{TEXT}
TEXT

\conclusions  %% \conclusions[modified heading if necessary]
TEXT




\appendix
\section{}    %% Appendix A

\subsection{}                               %% Appendix A1, A2, etc.


\authorcontribution{TEXT}

\begin{acknowledgements}
TEXT
\end{acknowledgements}


%% REFERENCES

%% The reference list is compiled as follows:

%%\begin{thebibliography}{}

%%\bibitem[AUTHOR(YEAR)]{LABEL}
%%REFERENCE 1

%%\bibitem[AUTHOR(YEAR)]{LABEL}
%%REFERENCE 2

%%\end{thebibliography}

%% Since the Copernicus LaTeX package includes the BibTeX style file copernicus.bst,
%% authors experienced with BibTeX only have to include the following two lines:
%%
\bibliographystyle{copernicus}
\bibliography{references}
%%
%% URLs and DOIs can be entered in your BibTeX file as:
%%
%% URL = {http://www.xyz.org/~jones/idx_g.htm}
%% DOI = {10.5194/xyz}


%% LITERATURE CITATIONS
%%
%% command                        & example result
%% \citet{jones90}|               & Jones et al. (1990)
%% \citep{jones90}|               & (Jones et al., 1990)
%% \citep{jones90,jones93}|       & (Jones et al., 1990, 1993)
%% \citep[p.~32]{jones90}|        & (Jones et al., 1990, p.~32)
%% \citep[e.g.,][]{jones90}|      & (e.g., Jones et al., 1990)
%% \citep[e.g.,][p.~32]{jones90}| & (e.g., Jones et al., 1990, p.~32)
%% \citeauthor{jones90}|          & Jones et al.
%% \citeyear{jones90}|            & 1990



%% FIGURES

%% ONE-COLUMN FIGURES

%%f
%\begin{figure}[t]
%\includegraphics[width=8.3cm]{FILE NAME}
%\caption{TEXT}
%\end{figure}
%
%%% TWO-COLUMN FIGURES
%
%%f
%\begin{figure*}[t]
%\includegraphics[width=12cm]{FILE NAME}
%\caption{TEXT}
%\end{figure*}
%
%
%%% TABLES
%%%
%%% The different columns must be seperated with a & command and should
%%% end with \\ to identify the column brake.
%
%%% ONE-COLUMN TABLE
%
%%t
%\begin{table}[t]
%\caption{TEXT}
%\begin{tabular}{column = lcr}
%\tophline
%
%\middlehline
%
%\bottomhline
%\end{tabular}
%\belowtable{} % Table Footnotes
%\end{table}
%
%%% TWO-COLUMN TABLE
%
%%t
%\begin{table*}[t]
%\caption{TEXT}
%\begin{tabular}{column = lcr}
%\tophline
%
%\middlehline
%
%\bottomhline
%\end{tabular}
%\belowtable{} % Table Footnotes
%\end{table*}
%
%
%%% NUMBERING OF FIGURES AND TABLES
%%%
%%% If figures and tables must be numbered 1a, 1b, etc. the following command
%%% should be inserted before the begin{} command.
%
%\addtocounter{figure}{-1}\renewcommand{\thefigure}{\arabic{figure}a}
%
%
%%% MATHEMATICAL EXPRESSIONS
%
%%% All papers typeset by Copernicus Publications follow the math typesetting regulations
%%% given by the IUPAC Green Book (IUPAC: Quantities, Units and Symbols in Physical Chemistry,
%%% 2nd Edn., Blackwell Science, available at: http://old.iupac.org/publications/books/gbook/green_book_2ed.pdf, 1993).
%%%
%%% Physical quantities/variables are typeset in italic font (t for time, T for Temperature)
%%% Indices which are not defined are typeset in italic font (x, y, z, a, b, c)
%%% Items/objects which are defined are typeset in roman font (Car A, Car B)
%%% Descriptions/specifications which are defined by itself are typeset in roman font (abs, rel, ref, tot, net, ice)
%%% Abbreviations from 2 letters are typeset in roman font (RH, LAI)
%%% Vectors are identified in bold italic font using \vec{x}
%%% Matrices are identified in bold roman font
%%% Multiplication signs are typeset using the LaTeX commands \times (for vector products, grids, and exponential notations) or \cdot
%%% The character * should not be applied as mutliplication sign
%
%
%%% EQUATIONS
%
%%% Single-row equation
%
%\begin{equation}
%
%\end{equation}
%
%%% Multiline equation
%
%\begin{align}
%& 3 + 5 = 8\\
%& 3 + 5 = 8\\
%& 3 + 5 = 8
%\end{align}
%
%
%%% MATRICES
%
%\begin{matrix}
%x & y & z\\
%x & y & z\\
%x & y & z\\
%\end{matrix}
%
%
%%% ALGORITHM
%
%\begin{algorithm}
%\caption{�}
%\label{a1}
%\begin{algorithmic}
%�
%\end{algorithmic}
%\end{algorithm}
%
%
%%% CHEMICAL FORMULAS AND REACTIONS
%
%%% For formulas embedded in the text, please use \chem{}
%
%%% The reaction environment creates labels including the letter R, i.e. (R1), (R2), etc.
%
%\begin{reaction}
%%% \rightarrow should be used for normal (one-way) chemical reactions
%%% \rightleftharpoons should be used for equilibria
%%% \leftrightarrow should be used for resonance structures
%\end{reaction}
%
%
%%% PHYSICAL UNITS
%%%
%%% Please use \unit{} and apply the exponential notation


\end{document}
