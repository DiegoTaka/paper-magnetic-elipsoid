%% Copernicus Publications Manuscript Preparation Template for LaTeX Submissions
%% ---------------------------------
%% This template should be used for copernicus.cls
%% The class file and some style files are bundled in the Copernicus Latex Package which can be downloaded from the different journal webpages.
%% For further assistance please contact the Copernicus Publications at: publications@copernicus.org
%% http://publications.copernicus.org


%% Please use the following documentclass and Journal Abbreviations for Discussion Papers and Final Revised Papers.


%% 2-Column Papers and Discussion Papers
\documentclass[gmd, manuscript]{copernicus}



%% Journal Abbreviations (Please use the same for Discussion Papers and Final Revised Papers)

% Archives Animal Breeding (aab)
% Atmospheric Chemistry and Physics (acp)
% Advances in Geosciences (adgeo)
% Advances in Statistical Climatology, Meteorology and Oceanography (ascmo)
% Annales Geophysicae (angeo)
% ASTRA Proceedings (ap)
% Atmospheric Measurement Techniques (amt)
% Advances in Radio Science (ars)
% Advances in Science and Research (asr)
% Biogeosciences (bg)
% Climate of the Past (cp)
% Drinking Water Engineering and Science (dwes)
% Earth System Dynamics (esd)
% Earth Surface Dynamics (esurf)
% Earth System Science Data (essd)
% Fossil Record (fr)
% Geographica Helvetica (gh)
% Geoscientific Instrumentation, Methods and Data Systems (gi)
% Geoscientific Model Development (gmd)
% Geothermal Energy Science (gtes)
% Hydrology and Earth System Sciences (hess)
% History of Geo- and Space Sciences (hgss)
% Journal of Sensors and Sensor Systems (jsss)
% Mechanical Sciences (ms)
% Natural Hazards and Earth System Sciences (nhess)
% Nonlinear Processes in Geophysics (npg)
% Ocean Science (os)
% Proceedings of the International Association of Hydrological Sciences (piahs)
% Primate Biology (pb)
% Scientific Drilling (sd)
% SOIL (soil)
% Solid Earth (se)
% The Cryosphere (tc)
% Web Ecology (we)
% Wind Energy Science (wes)


%% \usepackage commands included in the copernicus.cls:
%\usepackage[german, english]{babel}
%\usepackage{tabularx}
%\usepackage{cancel}
%\usepackage{multirow}
%\usepackage{supertabular}
%\usepackage{algorithmic}
%\usepackage{algorithm}
%\usepackage{amsthm}
%\usepackage{float}
%\usepackage{subfig}
%\usepackage{rotating}


\begin{document}

\title{3D MAGNETIC MODELLING FOR ELLIPSOIDS}


% \Author[affil]{given_name}{surname}

\Author[1]{Diego}{Takahashi Tomazella}
\Author[1]{Vanderlei}{C. Oliveira Jr}

\affil[1]{Department of Geophysics, Observatorio Nacional, Rio de Janeiro, Brazil}
%\affil[]{}

%% The [] brackets identify the author with the corresponding affiliation. 1, 2, 3, etc. should be inserted.



\runningtitle{TEXT}

\runningauthor{TEXT}

\correspondence{DIEGO TAKAHASHI TOMAZELLA (diego.takahashi@gmail.com)}



\received{}
\pubdiscuss{} %% only important for two-stage journals
\revised{}
\accepted{}
\published{}

%% These dates will be inserted by Copernicus Publications during the typesetting process.


\firstpage{1}

\maketitle



\begin{abstract}
In this work we will present results from a numerical modeling of the magnetic field and total field anomaly, represented by triaxial and prolate ellipsoidal sources. Such approach provide analytical results for anisotropy of magnetic susceptibility as well as for self-demagnetization effects, which can be easily adapted for distinctive geologic structures - hence being an useful tool for educational (e.g., potential methods and rock magnetism) and applied geophysics (e.g., characterization of high magnetic susceptibility, mineralized bodies) purposes. Numerical tests by means of a Python code (currently under development) allowed us to compared the effects of different geometries (ellipsoidal sources, spheres, dipole lines and elliptic cylinders), which were used to validate our computational implementation. This code will be freely available to the scientific community by the end of the year.
\end{abstract}



\introduction  %% \introduction[modified heading if necessary]

In the 19th century, \cite{dirichlet1839nouvelle} manage to solve the problem of the potential of a homogeneous ellipsoid with semi-axes $a \ge b \ge c$ by using the first and second Legendre's normal elliptic integrals. Later, \cite{maxwell1881treatise} showed that only bodies of second degree can be uniformly magnetized when placed in a uniform field. \cite{thomson1879and} presented the formulas of the potential and demagnetizing field using the standard integrals expressions for the ellispoid of revolution.

The formulas for the demagnetizing factors of the general ellipsoid alongside a table of values for different proportions of semi-axis was published by \cite{osborn1945demagnetizing}. In the same year, \cite{stoner1945xcvii} presented a more simple and friendly formulas of the demagnetizing factors for an ellipsoid, with step-by-step deduction. This work includes formulas for different parametrizations of the ellipsoid (triaxial, prolate, oblate).

A decade later, \cite{peake1953magnetic} published a paper with the formulas for the external field of ellipsoid of revolution. Some years after that, \cite{chang1961fields} showed the formulas for the external field of triaxial, prolate and oblate ellipsoids. Both used the Dirichlet solution in terms of elliptical integrals for external fields.
More recently, \cite{tejedor1995external}, gave the calculations of the external field generated by ellipsoids using a variational method, in contrast to the usual method of Dirichlet.

In the decade of 1970, \cite{farrar1979some} showed in his work the value of the ellipsoidal model in geophysic exploration, where it was used for analysis of magnetic anomalies in Tennant Creeks's gold mine in Australia. The ellipsoid was the most correct geometry to model bodies of lenticular pipe shape and the most appropriate to handle self-desmagnetization of bodies of high susceptibilty - magnetic susceptibility higher than 0.1 SI \citep{clark2014methods}.
\cite{guo1998self}, showed three case studies of the importance of self-demagnetization of high susceptibilties bodies when modelling magnetic sources. A magnetic-iron deposit, a volcanic-hosted iron deposit and an ultramafic-hosted nickel and copper deposits.
The same author years later,\cite{guo2001systematic}, published another paper analysing the errors of what an uncorrected by demagnetization data can affect in anomalies interpretation.

The magnetic field resultant of an ellipsoid was presented the first time in full modeling in the space domain by \cite{emerson1985magnetic}, along a compilation of several others geometric shapes. Through a consistent notation this work tapped a hole in the literature that missed the ellipsoidal model. This model, however, consists in a division between prolate and oblate ellipsoids, with a few differences in the algorithm.
A generalized model for triaxials ellipsoids was published by \cite{clark1986magnetic}. Both in Emerson et al. and Clark et al. works, they were the firsts to include the calculations of the magnetic field in a different coordinate system other than with the origin in the center of the ellipsoid (through a matrix of rotation). However, the codes presented for HP-41C calculators are outdated and don't have a descrition of the algorithm, difficulting the reproduction by the cientific commnunity. This may be the reason why the magnetic field modelling of the ellipsoid, in general, is manly used through commercial softwares.

As shown, the problem of the magnetic field of the ellipsoid is scattered through the literature, wich can difficult the access to many. This article come as a junction of previous works, making the problem easier and more understandable. Together we present an algorithm in a modern language with a complete descrition of it. The code will be available for free usage of the geophysics community.

\section{Methodology}

The total-field anomaly can be described as a difference between the total field vector and the induced magnetization of the crust of the Earth, that includes any anomalous distribuition of magnetization that can occur by magnetizied bodies in subsurface. This difference can be written as:
\begin{equation} \label{eq:1}
\Delta \textbf{T}^0_i = ||\textbf{T}_i|| - ||\textbf{F}_i||,
\end{equation}
where $\Delta \textbf{T}^0_i$ is the observed vector of total-field anomaly in the $i$-th position ($x_i$, $y_i$, $z_i$), $i = 1,...N$, $F_i$ is the geomagnetic field and $||.||$ is the Euclidian norm. The total-field vector $T_i$ is:
\begin{equation}\label{eq:2}
\textbf{T}_i = \textbf{F}_i + \textbf{B}_i,
\end{equation}
and $B_i$ is the total induced magnetization vector produced by all sources of anomalous susceptibility distribuition.
Considering $F_i$ a constant vector, $F_0$ for local ou regional scale, and that $||F_0|| \gg ||B_0||$, since, $B_i$ is a small pertubation of the magnetic field $F_i$, we can approximate the euclidian norm of the vector $T_i$ by a Taylor's expansion of first order:
\begin{equation}\label{eq:3}
||\textbf{T}_i|| \approx ||\textbf{F}_0 + \textbf{B}_i|| \approx ||\textbf{F}_0|| + \textbf{F}^T \textbf{B}_i
\end{equation}
where $T$ indicates transposition and
\begin{equation}\label{eq:4}
\textbf{F} = \dfrac{\textbf{F}_o}{||\textbf{F}_0||}
\end{equation}

is a unit vector that represents the gradient of the function $||T_i||$ in relation to the vector's components $T_i$. This way, we can approximate the Eq.(\ref{eq:1}) of total-field anomaly to:
\begin{equation}\label{eq:5}
\Delta T \approx \textbf{F}^T \textbf{B}_i, \quad i=1,...,NF
\end{equation}

The field now $\textbf{B}_i$ is represented this way to show that is in the geographic coordinates. One of the main steps to calculate the magnetic field of an ellipsoid is to change this system of coordinates to the body coordinates with origin in the center of it. The conversion can be done through a matrix of rotation $\textbf{V}$ that differs with the geometric type of ellipsoid, so the formulas will be shown later, separately for each one. But, in a general way, the conversion is:
\begin{equation}\label{eq:6}
\textbf{r}_i = \textbf{\textbf{V}} \tilde{\textbf{r}}_i + \textbf{r}_c,
\end{equation}

The vector $\textbf{r}_i$ has the positions $x,y,z$ in the geographic coordinates, $\tilde{\textbf{r}}_i$ has the positions $x1,x2,x3$ in the body coordinates and $\textbf{r}_c$ is the position of the center of the body in the geographic coordinates. The columns of $\textbf{\textbf{V}}$ are the eigenvectors that represents the directions cosines of the semi axes of the ellipsoid:
\begin{equation}
\textbf{\textbf{V}} =
\left[\begin{matrix} 
l_1 & l_2 & l_3 \\
m_1 & m_2 & m_3 \\
n_1 & n_2 & n_3
\end{matrix}\right] = 
\left[\begin{matrix} 
\textbf{v}_1 \\
\textbf{v}_2 \\
\textbf{v}_3
\end{matrix}\right]
\end{equation}


The equation of the surface of an ellipsoid can be written in a matrix form as:
\begin{equation}
(\textbf{r} - \textbf{r}_c)^T \textbf{A} (\textbf{r} - \textbf{r}_c) = 1,
\end{equation}

and $\textbf{\textbf{A}}$ is a 3X3 matrix formed by:
\begin{equation}
\textbf{\textbf{A}} = \textbf{\textbf{V}} 
\left[\begin{matrix} 
\dfrac{1}{a^2+s} & 0 & 0 \\
0 & \dfrac{1}{b^2+s} & 0 \\
0 & 0 & \dfrac{1}{c^2+s}
\end{matrix}\right]
\textbf{\textbf{V}}^T,
\end{equation}

The diagonal of the middle matrix, the eigenvalues that represents the reciprocals of the square of the semi-axes $a^{-2}$, $b^{-2}$ and $c^{-2}$.

For all types of ellipsoids the magnetic field can be written as:
\begin{equation}\label{eq:campoGeoNL}
\tilde{\textbf{B}}_i = \sum_{j=1}^{L} \tilde{\textbf{b}}_i^j, \quad i=1,...,N
\end{equation}

where $\textbf{b}_i^j$ is:
\begin{equation}\label{eq:campoGeo}
\tilde{\textbf{b}}_i^j = 2\pi a_j b_j c_j \times [\textbf{M}_i^j - \textbf{D}_i^j] \times \tilde{\textbf{J}}^j \quad j=1,...,L
\end{equation}

The matrices $\textbf{M}_i^j$ and $\textbf{D}_i^j$ are different for each one, therefore they will be shown separately in the next sections.

The same way the matrix of conversion can be used to change the magnetic field $\tilde{\textbf{B}}_i$ calculated in the body coordinates, back to the geopraphic, where is more useful to us:
\begin{equation}\label{eq:7}
\textbf{B}_i = \textbf{V} \tilde{\textbf{B}}_i
\end{equation}

Leading to:
\begin{equation}
\Delta \textbf{B}_x = \Delta \tilde{\textbf{B}}_1l_1 + \Delta \tilde{\textbf{B}}_2l_2 + \Delta \tilde{\textbf{B}}_3l_3
\end{equation}
\begin{equation}
\Delta \textbf{B}_y = \Delta \tilde{\textbf{B}}_1m_1 + \Delta \tilde{\textbf{B}}_2m_2 + \Delta \tilde{\textbf{B}}_3m_3
\end{equation}
\begin{equation}
\Delta \textbf{B}_z = \Delta \tilde{\textbf{B}}_1n_1 + \Delta \tilde{\textbf{B}}_2n_2 + \Delta \tilde{\textbf{B}}_3n_3
\end{equation}

\subsection{The forward model of a triaxial ellipsoid}

The implementation of the forward problem of a triaxial ellipsoid's magnetic field (three semi-axis $a > b > c$) is done in a new coordinate system, where its origin is the center of this ellipsoidal body.

This new coordinate system ($x_1,x_2,x_3$) is defined by the unit vectors $\textbf{v}_h (h=1,2,3)$ with the respect with the geographic axis $x$, $y$ e $z$:
\begin{equation}
\textbf{v}_1 = (l_1,l_2,l_3) = (-\cos\alpha \cos\delta, \cos\alpha \cos\gamma \sin\delta + \sin\alpha \sin\gamma, \sin\alpha \cos\gamma - \cos\alpha \sin\gamma \sin\delta)
\end{equation}

\begin{equation}
\textbf{v}_2 = (m_1,m_2,m_3) = (-\sin\alpha \cos\delta, \sin\alpha \cos\gamma \sin\delta - \cos\alpha \sin\gamma, -\cos\alpha \cos\gamma - \sin\alpha \sin\gamma \sin\delta)
\end{equation}

\begin{equation}
\textbf{v}_3 = (n_1,n_2,n_3) = (-\sin\delta,-\cos\gamma \cos\delta, \sin\gamma \cos\delta)
\end{equation}

The angles referring to the unit vectors are determined by the orientations of the ellipsoid's semi-axis. The angle $\alpha$ is the azimuth of semi-major axe ($a$) plus 180$^{\circ}$. While $\delta$ is the inclination of semi-major axe ($a$) in relation to the geographic plane. Lastly, $\gamma$ is the angle between the semi-mid axe ($b$) and the vertical projection of the ellipsoid's center with the geographic plane.

Thus, the body's coordinates are given by:

\begin{equation} \label{eq:bodyco}
x_h = (x-xc)l_h + (y-yc)m_h + (z-zc)n_h \quad (h= 1,2,3)
\end{equation}

Where $x_c$, $y_c$ e $z_c$ are the coordinates of the ellipsoid's center in the geographic system $x$, $y$ e $z$.

For an ellipsoid of semi-axis $a > b > c$, the equation that defines your surface is:

\begin{equation} \label{eq:ellipsoide}
\dfrac{x_1^2}{(a^2+s)} + \dfrac{x_2^2}{(b^2+s)} + \dfrac{x_3^2}{(c^2+s)} = 1
\end{equation}

The parameter $s$ controls the ellipsoid form. When $s$ gets close to $\infty$ the equation \eqref{eq:ellipsoide} tends to the sphere equation of radius $r=\sqrt{\lambda}$. When $s = -c^2$, the last term of the ellipsoid's equation is less than zero and it becomes the equation of a circle.

There is, however, a set of values for $s (\lambda,\mu,\nu)$, which are roots of the cubic equation:

\begin{equation} \label{eq:cubica}
s^3 + p_2s^2 + p_1s + p_0 = 0
\end{equation}

This set of roots, called ellipsoidal coordinates, correspond to the parameters of a point ($x_1,x_2,x_3$) which are under the intersection of three ortogonal surfaces related to the body coordinates. Their expressions are:

\begin{equation}
\lambda = 2 \sqrt{\left (\dfrac{-p}{3} \right)} \cos \left(\dfrac{\theta}{3} \right) - \dfrac{p_2}{3}
\end{equation}

\begin{equation}
\mu = -2 \sqrt{\left (\dfrac{-p}{3} \right)} \cos \left(\dfrac{\theta}{3} + \dfrac{\pi}{3} \right) - \dfrac{p_2}{3}
\end{equation}

\begin{equation}
\mu = -2 \sqrt{\left (\dfrac{-p}{3} \right)} \cos \left(\dfrac{\theta}{3} - \dfrac{\pi}{3} \right) - \dfrac{p_2}{3}
\end{equation}

Where:
\begin{equation}
p_0 = a^2b^2c^2 - b^2c^2x_1^2 - c^2a^2x_2^2 - a^2b^2x_3^2
\end{equation}

\begin{equation}
p_1 = a^2b^2 + b^2c^2 + c^2a^2 - (b^2+c^2)x_1^2 - (c^2+a^2)x_2^2 - (a^2+b^2)x_3^2
\end{equation}

\begin{equation}
p_2 = a^2 + b^2 + c^2 - x_1^2 - x_2^2 - x_3^2
\end{equation}

\begin{equation}
p = p_1 - \dfrac{p_2^2}{3}
\end{equation}

\begin{equation}
q = p_0 - \dfrac{p_1p_2}{3} + 2 \left(\dfrac{p_2}{3} \right)^3
\end{equation}

\begin{equation}
\theta = \cos^{-1} \left[\dfrac{-q}{2} \sqrt{\left(\dfrac{-p}{3} \right)^3} \right]
\end{equation}


The calculation of the largest root $\lambda$ of the equation \eqref{eq:cubica} is essential, since the magnetic field depends on the spatial derivatives of the equation \ref{eq:ellipsoide}, where $s$ admits the value of $\lambda$.

\begin{equation}
\dfrac{\partial \lambda}{\partial x_1} = \dfrac{2x_1/(a^2+\lambda)}{\left(\dfrac{x_1}{a^2+\lambda}\right)^2 + \left(\dfrac{x_2}{b^2+\lambda}\right)^2 + \left(\dfrac{x_3}{c^2+\lambda}\right)^2}
\end{equation}

\begin{equation}
\dfrac{\partial \lambda}{\partial x_2} = \dfrac{2x_2/(b^2+\lambda)}{\left(\dfrac{x_1}{a^2+\lambda}\right)^2 + \left(\dfrac{x_2}{b^2+\lambda}\right)^2 + \left(\dfrac{x_3}{c^2+\lambda}\right)^2}
\end{equation}

\begin{equation}
\dfrac{\partial \lambda}{\partial x_3} = \dfrac{2x_3/(c^2+\lambda)}{\left(\dfrac{x_1}{a^2+\lambda}\right)^2 + \left(\dfrac{x_2}{b^2+\lambda}\right)^2 + \left(\dfrac{x_3}{c^2+\lambda}\right)^2}
\end{equation}\\

The demagnetization factors are given by:

\begin{equation}
N_1 = \dfrac{4 \pi abc}{(a^2-b^2)\sqrt{(a^2-c^2)}} [F(\theta,k) - E(\theta,k)]
\end{equation}

\begin{equation}
N_2 = \dfrac{4 \pi abc \sqrt{(a^2-c^2)}}{(a^2-b^2)(b^2-c^2)} \left[E(\theta,k) - \left(\dfrac{b^2-c^2}{a^2-c^2} \right) F(\theta,k) - \dfrac{c(a^2-b^2)}{ab \sqrt{(a^2-c^2)}} \right]
\end{equation}

\begin{equation}
N_3 = \dfrac{4 \pi abc}{(b^2-c^2)\sqrt{(a^2-c^2)}} \left[\dfrac{b \sqrt{(a^2-c^2)}}{ac} - E(\theta,k) \right]
\end{equation}

Where $F(\theta,k)$ e $E(\theta,k)$ are first and second Legendre's normal eliptical integrals, respectively. To calculate $k$ and $\theta$ we have the following expressions:

\begin{equation}
k = \sqrt{\left(\dfrac{a^2-b^2}{a^2-c^2} \right)}
\end{equation}

\begin{equation}
\theta = \cos^{-1}(c/a) \quad (0 \le \theta \le \pi/2)
\end{equation}

The susceptibility tensor matrix is:

\begin{equation}
k_{ij} = \sum_r k_r (L_r l_i + M_r m_i + N_r n_i) (L_r l_j + M_r m_j + N_r n_j) \quad (r = 1,2,3)
\end{equation}

The Earth's field vector, $F$, and the remanent magnetization ,$J_N$, also must be converted to the body's coordinates:

\begin{equation}
\textbf{F}_i = F(l \, l_i + m \, m_i + n \, n_i)
\end{equation}

\begin{equation}
(\textbf{J}_N)_i = J_N (l_N l_i + m_n m_i + n_N n_i)
\end{equation}

Note that $L_r, M_r, N_r$ ($r = 1,2,3$), $l,m,n$ and $l_N, m_N, n_N$ depends on their respective angles of inclinations and declinations:
\begin{equation}
L_R,l,l_N = \cos(dec) \, \cos(inc)
\end{equation}

\begin{equation}
M_R,m,m_N = \sin(dec) \, \cos(inc)
\end{equation}

\begin{equation}
N_R,n,n_N = \sin(inc)
\end{equation}

In the case that the body has a very low susceptibility ($\chi < 0.1$ SI) the self-demagnatization is negligible and the resultant magnetic vector is given by:

\begin{equation}
\textbf{J}_R = \textbf{K}\textbf{F} + \textbf{J}_{NRM}
\end{equation}

For values bigger than 0.1 SI the resultant magnetic vector is:

\begin{equation}
\textbf{J}_{Rc} = \textbf{G}^{-1} \textbf{J}_R
\end{equation}

Where:

\begin{equation}
\textbf{G} = I + KN = \left[\begin{matrix} 
1+k_{11}N_1 \quad & k_{12}N_2 \quad & k_{13}N_3 \\\\
k_{21}N_1 \quad & 1+k_{22}N_2 \quad & k_{23}N_3 \\\\
k_{31}N_1 \quad & k_{23}N_2 \quad & 1+k_{33}N_3
\end{matrix}\right]
\end{equation}

This way, the components of the magnetic field produced by a triaxial ellipsoid in the body's coordinates outside of it is:

\begin{equation} \label{eq:campo_e}
\Delta \textbf{B}_1 = f_1 \dfrac{\partial \lambda}{\partial x_1} - 2 \pi abc \, J_1 \, A(\lambda) 
\end{equation}

\begin{equation}
\Delta \textbf{B}_2 = f_1 \dfrac{\partial \lambda}{\partial x_2} - 2 \pi abc \, J_2 \, B(\lambda)
\end{equation}

\begin{equation}
\Delta \textbf{B}_3 = f_1 \dfrac{\partial \lambda}{\partial x_3} - 2 \pi abc \, J_3 \, C(\lambda)
\end{equation}

where:

\begin{equation}
f_1 = \dfrac{2\pi abc}{\sqrt{[(a^2+\lambda)(b^2+\lambda)(c^2+\lambda)]}} \left[\dfrac{J_1x_1}{a^2+\lambda} + \dfrac{J_2x_2}{b^2+\lambda} + \dfrac{J_3x_3}{c^2+\lambda}\right]
\end{equation}

\begin{equation}
A(\lambda) = \dfrac{2}{(a^2-b^2) \sqrt{(a^2-c^2)}} [F(\theta^{'},k) - E(\theta^{'},k)]
\end{equation}

\begin{equation}
B(\lambda) = \dfrac{2 \sqrt{(a^2-c^2)}}{(a^2-b^2)(b^2-c^2)} \left[ E(\theta^{'},k) - \left(\dfrac{b^2-c^2}{a^2-c^2} \right) F(\theta^{'},k) - \dfrac{k^2\sin\theta^{'}\cos\theta^{'}}{\sqrt{(1-k^2\sin^2\theta^{'})}} \right]
\end{equation}

\begin{equation} \label{eq:integralC}
C(\lambda) = \dfrac{2}{(b^2-c^2) \sqrt{(a^2-c^2)}} \left[\dfrac{\sin\theta^{'} \sqrt{(1-k^2\sin^2\theta^{'})}} {\cos\theta^{'}} - E(\theta^{'},k) \right]
\end{equation}

And:

\begin{equation}
\theta^{'} = \sin^{-1} \left(\dfrac{a^2-c^2}{a^2+\lambda}\right)^{0.5} \quad (0 \le \theta^{'} \le \pi/2) 
\end{equation}

Both $F(\theta^{'}, k)$ as $E(\theta^{'}, k)$ are again the first and second  noraml Legendre's eliptical integrals. While $A(\lambda), B(\lambda), C(\lambda)$ are analytic solutions of the integrals of the potential equation of an ellipsoid. This problem was solved by Dirichlet in 1839 \citep{clark1986magnetic} for the gravitational potential given by:

\begin{equation}
U_i(x_1,x_2,x_3) = \pi abc G \rho \int_{0}^{\infty} \left[1 - \dfrac{x_1^2}{a^2+u} - \dfrac{x_2^2}{b^2+u} - \dfrac{x_3^2}{c^2+u}\right] \dfrac{du}{R(u)}
\end{equation}

That can be rewritten as:
\begin{equation}
U_i(x_1,x_2,x_3) = \pi abc G \rho \int_{0}^{\infty} [D(\lambda) - A(\lambda) - B(\lambda) - C(\lambda)] \dfrac{du}{R(u)}
\end{equation}

Where $G$ is the gravitational constant and $\rho$ the body's density.

Using the equations from \eqref{eq:campo_e} to \eqref{eq:integralC}, we can rearrange them in a matrix format that will be more computationally efficient. Thus, the magnetic field generated by the triaxial ellipsoid \eqref{eq:campoGeo} is:

\begin{equation}
\textbf{b}_i^j = 2\pi a_j b_j c_j \times [\textbf{M}_i^j - \textbf{D}_i^j] \times \textbf{J}^j \quad j=1,...,L
\end{equation}

With:

\begin{equation}
\textbf{M}_i^j = \dfrac{1}{\sqrt{[(a^2+\lambda)(b^2+\lambda)(c^2+\lambda)]}}
\left[\begin{matrix} 
\dfrac{\partial \lambda}{\partial x_1} \dfrac{x_1}{a^2+\lambda} \quad & \dfrac{\partial \lambda}{\partial x_1} \dfrac{x_2}{b^2+\lambda} \quad & \dfrac{\partial \lambda}{\partial x_1} \dfrac{x_3}{c^2+\lambda} \\\\
\dfrac{\partial \lambda}{\partial x_2} \dfrac{x_1}{a^2+\lambda} \quad & \dfrac{\partial \lambda}{\partial x_2} \dfrac{x_2}{b^2+\lambda} \quad & \dfrac{\partial \lambda}{\partial x_2} \dfrac{x_3}{c^2+\lambda} \\\\
\dfrac{\partial \lambda}{\partial x_3} \dfrac{x_1}{a^2+\lambda} \quad & \dfrac{\partial \lambda}{\partial x_3} \dfrac{x_2}{b^2+\lambda} \quad & \dfrac{\partial \lambda}{\partial x_3} \dfrac{x_3}{c^2+\lambda}
\end{matrix}\right]_{3 \times 3}
\end{equation}

\begin{equation}
\textbf{D}_i^j =
\left[\begin{matrix} 
A(\lambda) & 0 & 0 \\
0 & B(\lambda) & 0 \\
0 & 0 & C(\lambda)
\end{matrix}\right]
\end{equation}

Important to remenber that all this calcuations are in the body's coordinates. In order to transform to the observations points use equation \eqref{eq:7}.

\subsection{The forward model of a prolate ellipsoid}

The same way that the triaxial ellipsoid, the magnetic field generated by a prolate ellipsoid ($a$ > $b$), must be calculated in the body's coordinate. In this case the matrix $\textbf{V}$ is composed by the same equations shown for the triaxial, but considering gamma equal to $0^o$ :
\begin{equation}
\textbf{v}_1 = (l_1,l_2,l_3) = (-\cos\alpha \cos\delta, \cos\alpha \sin\delta, \sin\alpha)
\end{equation}

\begin{equation}
\textbf{v}_2 = (m_1,m_2,m_3) = (-\sin\alpha \cos\delta, \sin\alpha \sin\delta, -\cos\alpha)
\end{equation}

\begin{equation}
\textbf{v}_3 = (n_1,n_2,n_3) = (-\sin\delta, -\cos\delta, 0)
\end{equation}

The coordinate transformation is the same as equation \eqref{eq:bodyco}.
The calculation of $\lambda$ is and its spatial derivatives are:
\begin{equation}
\lambda = \dfrac{r^2-a^2-b^2+ \Delta}{2},
\end{equation}

where:
\begin{equation}
r = \sqrt{x^2_1 + x^2_2 + x^2_3},
\end{equation}

and
\begin{equation}
\Delta = \sqrt{r^4 - (a^2-b^2)^2 -2(a^2-b^2)(x^2_1 - x^2_2 - x^2_3)},
\end{equation}

Hence its spatial derivatives are:
\begin{equation}
\dfrac{\partial \lambda}{\partial x_1} = x_1 \left(1 + \dfrac{r^2-a^2+b^2}{\Delta}\right)
\end{equation}

\begin{equation}
\dfrac{\partial \lambda}{\partial x_2} = x_2 \left(1 + \dfrac{r^2+a^2-b^2}{\Delta}\right)
\end{equation}

\begin{equation}
\dfrac{\partial \lambda}{\partial x_3} = x_3 \left(1 + \dfrac{r^2+a^2-b^2}{\Delta}\right)
\end{equation}

The demagnetization factor are:
\begin{equation}
N_1 = \dfrac{4\pi ab^2}{(a^2-b^2)^{3/2}} \left\{ \log_e \left[\left(\sqrt{\dfrac{a^2}{b^2}-1}\right) + \dfrac{a}{b}\right] - \sqrt{\left(1 - \dfrac{b^2}{a^2}\right)} \right\}
\end{equation}

\begin{equation}
N_2 = 2\pi - \dfrac{N_1}{2}
\end{equation}

The calculations of the susceptibility tensor $k$, Earth's field $\textbf{F}_i$, remanent magnetization vector $(\textbf{J}_N)_i$ and the resultant magnetization vector $\textbf{J}_{Rc}$ are the same as the triaxial ellipsoid. 

The magnetic field generated by the prolate ellipsoid \eqref{eq:campoGeo} can also be written in a matrix form as:

\begin{equation}
\textbf{b}_i^j = 2\pi a_j b_j c_j \times [\textbf{M}_i^j - \textbf{D}_i^j] \times \textbf{J}^j \quad j=1,...,L
\end{equation}

With:

\begin{equation}
\textbf{M}_i^j = \dfrac{1}{(a^2+\lambda)^{1/2}(b^2+\lambda)}
\left[\begin{matrix} 
\dfrac{\partial \lambda}{\partial x_1} \dfrac{x_1}{a^2+\lambda} \quad & \dfrac{\partial \lambda}{\partial x_1} \dfrac{x_2}{b^2+\lambda} \quad & \dfrac{\partial \lambda}{\partial x_1} \dfrac{x_3}{b^2+\lambda} \\\\
\dfrac{\partial \lambda}{\partial x_2} \dfrac{x_1}{a^2+\lambda} \quad & \dfrac{\partial \lambda}{\partial x_2} \dfrac{x_2}{b^2+\lambda} \quad & \dfrac{\partial \lambda}{\partial x_2} \dfrac{x_3}{b^2+\lambda} \\\\
\dfrac{\partial \lambda}{\partial x_3} \dfrac{x_1}{a^2+\lambda} \quad & \dfrac{\partial \lambda}{\partial x_3} \dfrac{x_2}{b^2+\lambda} \quad & \dfrac{\partial \lambda}{\partial x_3} \dfrac{x_3}{b^2+\lambda}
\end{matrix}\right]_{3 \times 3}
\end{equation}

\begin{equation}
\textbf{D}_i^j =
\left[\begin{matrix} 
A(\lambda) & 0 & 0 \\
0 & B(\lambda) & 0 \\
0 & 0 & B(\lambda)
\end{matrix}\right],
\end{equation}

where:
\begin{equation}
A(\lambda) = \dfrac{2}{(a^2-b^2)^{3/2}} \left\{\sqrt{\dfrac{a^2-b^2}{a^2+ \lambda}} - \log_e[\dfrac{(a^2-b^2)^{1/2} + (a^2+\lambda)^{1/2}}{(b^2+\lambda)^{1/2}}]  \right\}
\end{equation}

\begin{equation}
B(\lambda) = \dfrac{1}{(a^2-b^2)^{3/2}} \left\{\dfrac{\sqrt{(a^2-b^2)(a^2+\lambda)}}{b^2+\lambda} - \log_e[\dfrac{(a^2-b^2)^{1/2} + (a^2+\lambda)^{1/2}}{(b^2+\lambda)^{1/2}}]  \right\}
\end{equation}

\subsection{The forward model of an oblate ellipsoid}

The magnetic field generated by an oblate ellipsoid ($b$ > $a$), also must be calculated in the body's coordinate. The matrix $\textbf{V}$ is composed by the same equations shown for the prolate ellipsoid, but doing the changes due to the major semi-axe being $b$ :
\begin{equation}
\textbf{v}_1 = (l_1,l_2,l_3) = (\cos\alpha \sin\delta, -\cos\alpha \cos\delta, -\sin\alpha)
\end{equation}

\begin{equation}
\textbf{v}_2 = (m_1,m_2,m_3) = (\sin\alpha \sin\delta, -\sin\alpha \cos\delta, \cos\alpha)
\end{equation}

\begin{equation}
\textbf{v}_3 = (n_1,n_2,n_3) = (-\cos\delta, -\sin\delta, 0)
\end{equation}

The coordinate transformation is the same as equation \eqref{eq:bodyco}.
The calculation of $\lambda$ is and its spatial derivatives are the same as the prolate case.

The demagnetization factor are:
\begin{equation}
N_1 = \dfrac{4\pi ab^2}{(b^2-a^2)^{3/2}} \left[\dfrac{\sqrt{b^2-a^2}}{a} - \tan^{-1} \dfrac{\sqrt{b^2-a^2}}{a}\right]
\end{equation}

\begin{equation}
N_2 = 2\pi - \dfrac{N_1}{2}
\end{equation}

The calculations of the susceptibility tensor $k$, Earth's field $\textbf{F}_i$, remanent magnetization vector $(\textbf{J}_N)_i$ and the resultant magnetization vector $\textbf{J}_{Rc}$ are the same as the triaxial ellipsoid. 

The magnetic field generated by the prolate ellipsoid \eqref{eq:campoGeo} can also be written in a matrix form as:

\begin{equation}
\textbf{b}_i^j = 2\pi a_j b_j c_j \times [\textbf{M}_i^j - \textbf{D}_i^j] \times \textbf{J}^j \quad j=1,...,L
\end{equation}

With:

\begin{equation}
\textbf{M}_i^j = \dfrac{1}{(a^2+\lambda)^{1/2}(b^2+\lambda)}
\left[\begin{matrix} 
\dfrac{\partial \lambda}{\partial x_1} \dfrac{x_1}{a^2+\lambda} \quad & \dfrac{\partial \lambda}{\partial x_1} \dfrac{x_2}{b^2+\lambda} \quad & \dfrac{\partial \lambda}{\partial x_1} \dfrac{x_3}{b^2+\lambda} \\\\
\dfrac{\partial \lambda}{\partial x_2} \dfrac{x_1}{a^2+\lambda} \quad & \dfrac{\partial \lambda}{\partial x_2} \dfrac{x_2}{b^2+\lambda} \quad & \dfrac{\partial \lambda}{\partial x_2} \dfrac{x_3}{b^2+\lambda} \\\\
\dfrac{\partial \lambda}{\partial x_3} \dfrac{x_1}{a^2+\lambda} \quad & \dfrac{\partial \lambda}{\partial x_3} \dfrac{x_2}{b^2+\lambda} \quad & \dfrac{\partial \lambda}{\partial x_3} \dfrac{x_3}{b^2+\lambda}
\end{matrix}\right]_{3 \times 3}
\end{equation}

\begin{equation}
\textbf{D}_i^j =
\left[\begin{matrix} 
A(\lambda) & 0 & 0 \\
0 & B(\lambda) & 0 \\
0 & 0 & B(\lambda)
\end{matrix}\right],
\end{equation}

where:
\begin{equation}
A(\lambda) = \dfrac{2}{(b^2-a^2)^{3/2}} \left\{\sqrt{\dfrac{b^2-a^2}{a^2+ \lambda}} - \tan^{-1}\sqrt{\dfrac{b^2-a^2}{a^2+\lambda}}  \right\}
\end{equation}

\begin{equation}
B(\lambda) = \dfrac{1}{(b^2-a^2)^{3/2}} \left\{\tan^{-1}\sqrt{\dfrac{b^2-a^2}{a^2+\lambda}} - \dfrac{\sqrt{(b^2-a^2)(a^2+\lambda)}}{b^2+\lambda}  \right\}
\end{equation}

\subsubsection{HEADING}
TEXT




\conclusions  %% \conclusions[modified heading if necessary]
TEXT




\appendix
\section{}    %% Appendix A

\subsection{}                               %% Appendix A1, A2, etc.


\authorcontribution{TEXT}

\begin{acknowledgements}
TEXT
\end{acknowledgements}


%% REFERENCES

%% The reference list is compiled as follows:

%%\begin{thebibliography}{}

%%\bibitem[AUTHOR(YEAR)]{LABEL}
%%REFERENCE 1

%%\bibitem[AUTHOR(YEAR)]{LABEL}
%%REFERENCE 2

%%\end{thebibliography}

\bibliographystyle{copernicus}
\bibliography{references}

%% Since the Copernicus LaTeX package includes the BibTeX style file copernicus.bst,
%% authors experienced with BibTeX only have to include the following two lines:
%%
%% \bibliographystyle{copernicus}
%% \bibliography{example.bib}
%%
%% URLs and DOIs can be entered in your BibTeX file as:
%%
%% URL = {http://www.xyz.org/~jones/idx_g.htm}
%% DOI = {10.5194/xyz}


%% LITERATURE CITATIONS
%%
%% command                        & example result
%% \citet{jones90}|               & Jones et al. (1990)
%% \citep{jones90}|               & (Jones et al., 1990)
%% \citep{jones90,jones93}|       & (Jones et al., 1990, 1993)
%% \citep[p.~32]{jones90}|        & (Jones et al., 1990, p.~32)
%% \citep[e.g.,][]{jones90}|      & (e.g., Jones et al., 1990)
%% \citep[e.g.,][p.~32]{jones90}| & (e.g., Jones et al., 1990, p.~32)
%% \citeauthor{jones90}|          & Jones et al.
%% \citeyear{jones90}|            & 1990



%% FIGURES

%% ONE-COLUMN FIGURES

%%f
%\begin{figure}[t]
%\includegraphics[width=8.3cm]{FILE NAME}
%\caption{TEXT}
%\end{figure}
%
%%% TWO-COLUMN FIGURES
%
%%f
%\begin{figure*}[t]
%\includegraphics[width=12cm]{FILE NAME}
%\caption{TEXT}
%\end{figure*}
%
%
%%% TABLES
%%%
%%% The different columns must be seperated with a & command and should
%%% end with \\ to identify the column brake.
%
%%% ONE-COLUMN TABLE
%
%%t
%\begin{table}[t]
%\caption{TEXT}
%\begin{tabular}{column = lcr}
%\tophline
%
%\middlehline
%
%\bottomhline
%\end{tabular}
%\belowtable{} % Table Footnotes
%\end{table}
%
%%% TWO-COLUMN TABLE
%
%%t
%\begin{table*}[t]
%\caption{TEXT}
%\begin{tabular}{column = lcr}
%\tophline
%
%\middlehline
%
%\bottomhline
%\end{tabular}
%\belowtable{} % Table Footnotes
%\end{table*}
%
%
%%% NUMBERING OF FIGURES AND TABLES
%%%
%%% If figures and tables must be numbered 1a, 1b, etc. the following command
%%% should be inserted before the begin{} command.
%
%\addtocounter{figure}{-1}\renewcommand{\thefigure}{\arabic{figure}a}
%
%
%%% MATHEMATICAL EXPRESSIONS
%
%%% All papers typeset by Copernicus Publications follow the math typesetting regulations
%%% given by the IUPAC Green Book (IUPAC: Quantities, Units and Symbols in Physical Chemistry,
%%% 2nd Edn., Blackwell Science, available at: http://old.iupac.org/publications/books/gbook/green_book_2ed.pdf, 1993).
%%%
%%% Physical quantities/variables are typeset in italic font (t for time, T for Temperature)
%%% Indices which are not defined are typeset in italic font (x, y, z, a, b, c)
%%% Items/objects which are defined are typeset in roman font (Car A, Car B)
%%% Descriptions/specifications which are defined by itself are typeset in roman font (abs, rel, ref, tot, net, ice)
%%% Abbreviations from 2 letters are typeset in roman font (RH, LAI)
%%% Vectors are identified in bold italic font using \vec{x}
%%% Matrices are identified in bold roman font
%%% Multiplication signs are typeset using the LaTeX commands \times (for vector products, grids, and exponential notations) or \cdot
%%% The character * should not be applied as mutliplication sign
%
%
%%% EQUATIONS
%
%%% Single-row equation
%
%\begin{equation}
%
%\end{equation}
%
%%% Multiline equation
%
%\begin{align}
%& 3 + 5 = 8\\
%& 3 + 5 = 8\\
%& 3 + 5 = 8
%\end{align}
%
%
%%% MATRICES
%
%\begin{matrix}
%x & y & z\\
%x & y & z\\
%x & y & z\\
%\end{matrix}
%
%
%%% ALGORITHM
%
%\begin{algorithm}
%\caption{�}
%\label{a1}
%\begin{algorithmic}
%�
%\end{algorithmic}
%\end{algorithm}
%
%
%%% CHEMICAL FORMULAS AND REACTIONS
%
%%% For formulas embedded in the text, please use \chem{}
%
%%% The reaction environment creates labels including the letter R, i.e. (R1), (R2), etc.
%
%\begin{reaction}
%%% \rightarrow should be used for normal (one-way) chemical reactions
%%% \rightleftharpoons should be used for equilibria
%%% \leftrightarrow should be used for resonance structures
%\end{reaction}
%
%
%%% PHYSICAL UNITS
%%%
%%% Please use \unit{} and apply the exponential notation


\end{document}
